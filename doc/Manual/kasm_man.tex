% Document Type: LaTeX
% Master File: kasm_man.tex
\documentstyle[10pt,fullpage,psfig]{article}

\newcommand{\ifsomething}[2]{%
{\setbox0=\hbox{#1}%
\ifdim\wd0=0pt%
\else%
{#2}\fi}}

\newsavebox{\kbitnames}
\newsavebox{\cbitnames}

\def\s#1#2#3{\put(#1,3.5){\makebox(0,0)[c]{{\tiny\rm #2}}}%
\put(#1,2.0){\makebox(0,0)[c]{\raisebox{0pt}[0pt][0pt]{\tiny #3}}}}

\savebox{\kbitnames}{\footnotesize\tt\setlength{\unitlength}{1.21ex}%
\begin{picture}(70,4)%
\s{1}{fr}{ce}\s{5}{alu}{func}\s{9}{c}{i}\s{11}{m}{p}%
\s{13}{l}{c}\s{15}{f}{i}\s{19}{flag}{bus}\s{24.5}{bit}{shift}%
\s{28.5}{res}{mx}\s{32}{sram}{}\s{31}{}{r}\s{33}{}{w}\s{36}{opB}{sel}%
\s{41.5}{opA}{register}\s{49}{opC}{register}\s{56}{dest}{register}%
\s{64}{array}{immediate}\end{picture}}

\savebox{\cbitnames}{\footnotesize\tt\setlength{\unitlength}{1.22ex}%
\begin{picture}(70,4)%
\s{1.5}{spa}{re}\s{4}{im}{mx}\s{7}{diag}{out}\s{10}{D}{pop}%
\s{20}{controller}{immediate}\s{29}{br}{k}\s{31}{f}{ot}%
\s{34}{i/o}{}\s{33}{}{r}\s{35}{}{w}%
\s{38}{dsel}{}\s{37}{}{r}\s{39}{}{l}%
\s{42.5}{scratch}{}\s{41}{}{w}\s{43.5}{}{mx}%
\s{46}{}{sh}\s{48}{}{ld}\s{47}{cbs}{}%
\s{50.5}{imm}{sel}\s{54}{br}{}\s{53}{}{w}\s{55}{}{c}%
\s{58}{cnt}{}\s{57}{}{d}\s{59}{}{ps}%
\s{63.5}{pc}{}\s{61}{}{po}\s{63}{}{pu}\s{65.5}{}{sel}%
\end{picture}}


\def\kbits#1{{\footnotesize\tt\setlength{\unitlength}{1.21ex}%
\begin{picture}(70,4)%
\put(0,0){\usebox{\kbitnames}}\put(0.5,0){\mbox{\tt #1}}\end{picture}}}

\def\cbits#1{{\footnotesize\tt\setlength{\unitlength}{1.22ex}%
\begin{picture}(70,4)%
\put(0,0){\usebox{\cbitnames}}%
\put(0.5,0){\mbox{\tt #1}}\end{picture}}}

\setlength{\parindent}{0pt}
\long\def\kasmdesc#1#2#3#4{%
\par\medskip{\bf Description:}\hspace{2ex} #1
\ifsomething{#2}{\par{\bf Comments:}\hspace{2ex} #2}
\ifsomething{#3}{\par\medskip\centerline{\kbits{#3}}}
\ifsomething{#4}{\par\medskip\centerline{\cbits{#4}}}
}
\pagestyle{headings}
% opcode, name, unused, operation, syntax, uses, sets
% {descr, comments, kestrel, controller}
\long\def\kasm#1#2#3#4#5#6#7#8#9{{%
\index{{\bf #1}}#9%
\protect\setlength{\parindent}{0pt}%
\begin{minipage}{\textwidth}
{\Large\bf #1} \hfill {\bf #2}\hfill {\Large\bf #1}\par\medskip\par
{\bf Operation:} {\sc {#4}}\hfill {\bf Uses:} {\sc {#6}}\\
{\bf Syntax: } {#5}
\hfill {\bf Sets:} {\sc {#7}}
\kasmdesc #8
\par\centerline{\rule{0.6\textwidth}{0.5pt}}
\end{minipage}\bigskip}}

\makeindex
\begin{document}

\vspace*{0.1\textheight}

{\large
\thispagestyle{empty}
\begin{center}
\Huge KASM Assembly Manual
\end{center}

\vspace{2em}
\begin{center}
\Large Richard Hughey\\~\\UCSC Kestrel Project\\
Department of Computer Engineering\\Jack Baskin School of Engineering\\
University of California\\Santa Cruz, CA
\end{center}

\bigskip
\begin{center}
Technical Report UCSC-CRL-98-11\\
September, 1998
\end{center}
 
\vspace{2em}

\hspace*{\fill}
\psfig{figure=kestrelbw.eps,width=0.2\textheight}
\hspace*{\fill}
\psfig{figure=mediumslug.ps,width=0.2\textheight}
\hspace*{\fill}

}

\clearpage
\tableofcontents
\clearpage

\section{Introduction}

This manual documents the opcodes, fields, and directives supported
by the new {\sc Kasm\/} assembler.  The manual assumes familiarity
with the Kestrel architecture.  The reader is referred to the papers
available from the http://www.cse.ucsc.edu/research/kestrel for more
information on the system architecture.

\section{Running Kasm}

The {\sc Kasm\/} command line is:
\begin{center}
\verb+newkasm [-g] [-b] [-Dsym=value] [-o outfile] file[.kasm]+
\end{center}
Where {\tt file} is a {\sc Kasm\/} assembly language file.  The
{\verb+-g+} option indicates that the output file should be annotated with
debugging information. The \verb+-b+ option, useful in debugging
assembly code, indicates that output 
should be in binary rather than the hexadecimal required by {\sc Kestrel}
The \verb+[-D]+ option can be used to define
one or more symbols as if they had {\tt define} statements in the
assembly file.  The {\sc Kasm\/} assembler will output 
Kestrel object code, and debugging information if requested, into
{\verb+file.ko+}.  The \verb+-o+ option indicates that a different
file should be used.  Error messages are printed to {\tt stderr} in a 
format compatible with Emacs' compile mode.

\section{Running Kestrel}

{\sc Kasm} programs are run in the Kestrel runtime environment.
The {\sc kestrel} command line is:
\begin{center}
 \verb+kestrel -[s|b] [-debug] objectfile.ko inputfile ouputfile [#procesors] +
\end{center}

Here, the selection of {\tt s} or {\tt b} picks either the simulator
or the Kestrel hardware.  Use of the Kestrel hardware requires that
one of the two Kestrel boards be available and that its socket-based
server be running.

The {\tt debug} option places the user immediately in the Kestrel
debugger.  A series of menus can be used to run or step the program
and examine the contents of PEs in either the simulator or the
hardware.

The object file is the output from the assembler, and the required
input and output files have the data for the program, which must be in
the order required by the programs.  When {\sc Kestrel} is used to
access the simulator, the number of processing elements can be
specified  (the default is \verb+#512+, to agree with the hardware).

\section{Kasm syntax}

{\sc Kasm\/} assembly code makes use of the opcodes, fields, and
directives specified below.  A typical {\sc Kasm\/} line includes one
or more compatible opcodes, a destination register, zero to three
Kestrel operands, and possibly some instruction modifiers or
controller commands.  Comments are delimited with semicolons ---
whenever a semicolon is included in a line, the semicolon and the
remainder of the line are ignored.

Opcodes, labels, defines, and macro names are all case insensitive.

Each opcode description includes its operation, syntax, a list (apart
from registers) of Kestrel internal state that is used and that is
changed,  description and comments, and when appropriate a bit
pattern.  The bit patterns define the behavior of the instruction in
the assembler.  Bit patters are made up of several single-letter
specifiers which are interpreted as follows:

\begin{center}
\begin{tabular}{ll}
 x    &Does not affect given bit\\
 0,1  &Bit set to 0 or 1\\
 z,w  &Bit defaults to 0 or 1 (can be overridden)\\
 A,B  &1/0 if opA is first operand, 0/1 otherwise (func bits only)\\
 a,b  &Default A,B (can be overridden)\\
 E    &Either opA or opB must be specified but not both (opA and opB
 bits only)\\
 e    &Default E (can be overridden)\\
 l    &Label required in Cimm\\
 r    &Required --- must be specified by another field\\
 u    &Unused bit, error if another field sets
\end{tabular}
\end{center}

Opcodes can generally be listed in any order on a line, except that
those that require operands should be listed first to ensure the
correct parsing of the operands.  For example, to perform an add-min
instruction, the code
\begin{center}
\tt add L1, L2, MDR, L3, minc
\end{center}
would generate an error message because the third source operand ({\tt L3})
is unneeded for an addition.  The correct form would place the {\tt
minc} after the {\tt add}.

It is suggested opcodes and fields that require operands or affect the
output of the ALU (such as {\tt mp}, the arithmetic multi-precision
indicator) be placed first on the line, followed by the operands,
followed by other modifiers, such  controller (for example, {\tt
jump}) and bitshifter control fields (for example, {\tt bspush}).

Comments, symbolic constants (see {\tt define}), and macros can be
used to make code more readable.

\input{kasmopc}

\def\indexentry#1#2{\item #1, #2}

\clearpage

\input kasm_man.ind



\end{document}
